\indent Diplomová práca nám ozrejmila pojmy ako virtuálne laboratórium, jeho jednotlivé komponenty a využitie v súčasnej dobe.

Cieľom tejto práce bolo naštudovať problematiku virtuálnych laboratórií, vytvoriť stručnú analýzu existujúcich riešení a vytvoriť komplexnú aplikáciu, ktorá sa mala skladať z viacerých častí. Začali sme tvorbou a úpravou referenčnej simulácie šikmého vrhu pre Simulink podľa potrieb. Ďalej bolo treba zabezpečiť prenos dát medzi Matlabom a Node.js, čo sme vyriešili vďaka RESTful komunikácií. Údaje bolo potrebné posielať ďalej do webového prehliadača. Táto časť sa vyriešila použiťím JavaScript knižnice Socket.io čo je obdoba websocketov s rozšírenou podporou aj pre staršie prehliadače. Na strane webu sa využil aktuálne populárny frontend JavaScript framework Angular.js. Medzi dôležité súčasti implementácie nebolo len zobrazovať realtime údaje v grafoch, ale ich aj ukladať pre neskoršie spracovanie. Údaje sa podarilo ukladať do dokumentovej databázy MongoDB.
Ciele, ktoré sme si stanovili, sme aj splnili.

Za konečným výsledkom je vidieť množstvo práce. Síce súčasné riešenie nie je možné nasadiť do reálnej prevádzky bez istých úprav a integrácií, ale poslúži ako solídny základ, na ktorom je možné stavať a využiť ho minimálne v priestoroch FEI STU na simuláciu systému, alebo na zber dát z reálneho zariadenia. Po vytvorení riešenia sme zhodnotili, že pri tvorbe virtuálneho laboratória na platforme Node.js bol vývoj jednoduchší vďaka využitiu JavaScriptu na serverovej aj klientskej strane. Mysleli sme si, že vďaka jednovláknovej slučke Node.js bude zvládať viacej klientov a simulácií, ako podobné riešenie na inej platforme. Nie je problém pri veľkom počte prihlásených užívateľov, ale pri spustení viacero simulácií v Matlabe. Na testovanom zariadení (Macbook Pro) bol problém už pri dvoch súčasných simuláciach. Riešenie vidíme pri použití výkonného servera pre Matlab výpočty.

Tu sa práca ešte nekončí a je možné rozsíriť StarkLab ďalšou zaujímavou funkcionalitou. Ako napríklad zovšeobecnenie rozhrania pre komunikáciu. Vytvorenie rozhrania aj pre iný software ako Matlab pre jednoduchšie pripojenie výpočtov, alebo simulácií. Nasadenie Matlabu na samostaný server s dostupnou doménou. Nahrávanie simulácií a Matlab súborov cez webové rozhranie. Možností ako vylepšiť toto riešenie je množstvo. V prípade pokračovania v tejto téme sa študent môže nimi inšpirovať.