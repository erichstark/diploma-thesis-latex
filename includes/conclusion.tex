\indent Diplomová práca nám ozrejmila pojmy ako virtuálne laboratórium, aké môže mať komponenty a jeho využitie v súčasnej dobe.\\
\indent Cieľom tejto práce bolo naštudovanie problematiku virtuálnych laboratórií, vytvoriť stručnú analýzu existujúcich riešení a vytvoriť komplexnú aplikáciu, ktorá sa mala skladať z viacerých častí. Začali sme tvorbou a úpravou referenčnej simulácie šikmého vrhu pre Simulink podľa potrieb. Ďalej bolo treba zabezpečiť prenos dát medzi Matlabom a Node.js, čo sme vyriešili vďaka RESTful komunikácií. Údaje bolo potrebné posielať ďalej k web klientom. Táto časť sa vyriešila použiťím JavaScript knižnice socket.io čo je obdoba websocketov s rozšírenou podporou aj pre staršie prehliadače. Na strane webu sa využil aktuálne populárny frontend JavaScript framework Angular.js. Medzi dôležité súčasti implementácie nebolo len zobrazovať realtime údaje v grafoch, ale ich aj ukladať pre neskoršie spracovanie. Údaje sa podarilo ukladať do dokumentovej databáze MongoDB.
Ciele, ktoré sme si stanovili, sme aj splnili.\\
\indent Za konečným výsledkom je vidieť mnoho práce. Síce súčasné riešenie nie je možné nasadiť do reálnej prevádzky bez istých úprav a integrácií, ale poslúži ako solídny základ, na ktorom je možné stavať a využiť ho minimálne v priestoroch FEI STU na simuláciu systému alebo na zber dát z reálneho zariadenia.