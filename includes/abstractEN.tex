Thesis deals with possibilities of using modern Node.js platform in virtual laboratories and create a reference application in combination with other technologies such as Matlab, Simulink, Angular.js and MongoDB. In the introduction we describe the characteristics of virtual laboratories and its possible components. We also discuss the possibilities of interaction with the experiment. In the next section, we consider existing solutions with their respective weaknesses. The technology and it's main characteristics are described in the following section. Implementation of solution was carried out by creating smaller parts. At first we have implemented a motion of projectile in Matlab and Simulink. It was necessary to get data from Simulink to Matlab workspace. Then we had to send them to Node.js using RESTful web services. On the side of Node.js was Socket.io waiting for the data and sent them to the web browser. The last part refers to the visualization of data in the browser in the form of graphs, animations, data table and subsequently write data into the database. The result of this thesis is a functional solution called StarkLab, where it is possible implement own simulation.