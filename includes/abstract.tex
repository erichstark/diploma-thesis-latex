Diplomová práca sa zaoberá využitím možností modernej platformy Node.js v oblasti virtuálnych laboratórií a vytvorením referenčnej aplikácie v spojení technológií Node.js, Matlab, Simulink, Angular.js a MongoDB. Úvod práce je venovaný základným informáciam o virtuálnych laboratóriach a jeho komponentoch. Taktiež zistujeme možnosti pracovania s experimentami. Ďalej je porovnanie už existujúcich riešení a ich možných nedostatkov v súčasnosti. Nasleduje sekcia, kde sme si definovali technológie a ich základne časti, ktoré plánujeme využívať. Implementácia riešenia prebiehala vytvorením menších častí. Najprv sme implementovali referenčnú simuláciu šikmého vrhu do Matlabu a Simulinku. Odtiaľ bolo potrebné získať generované údaje pomocou RESTful služieb do Node.js. Po ich odoslaní boli spracované a cez socket.io zaslané do prehliadača, kde sa zobrazili údaje do grafu, animácie, tabuľky a súčasne zapísali do databázy MongoDB.