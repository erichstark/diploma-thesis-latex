Diplomová práca sa zaoberá využitím možností modernej platformy Node.js v oblasti virtuálnych laboratórií a vytvorením referenčnej aplikácie v spojení ďalších technológií ako Matlab, Simulink, Angular.js a MongoDB. V úvode práce sú popísané vlastnosti virtuálnych laboratórií a jeho možných komponentov. Taktiež pojednávame o možnostiach interakcie s experimentami. V ďalšej časti boli porovnané už existujúce riešenia a ich možné nedostatky v súčasnosti. Nasleduje sekcia, kde sme si definovali technológie a ich hlavné vlastnosti, ktoré sme plánovali využiť. Implementácia riešenia prebiehala vytvorením menších častí. Ako prvú sme implementovali referenčnú simuláciu šikmého vrhu do Matlabu a Simulinku. Bolo potrebné získať dáta zo Simulinku do Matlab workspace. Ten ich následne posiela do Node.js pomocou RESTful služieb. Na strane Node.js čaká na dáta Socket.io, ktorý ich pošle do webového prehliadača. Posledná časť hovorí o vizualizácií dát v prehliadači vo forme grafu, animácie a tabuľky a následné zapísanie do databázy. Výsledkom práce je funkčné riešenie, kde je možné implementovať vlastnú simuláciu.